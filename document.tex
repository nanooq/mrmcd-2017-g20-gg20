\documentclass[]{beamer}
\usepackage[utf8]{inputenc}
\usepackage[T1]{fontenc}
\usepackage{lmodern}

%\setbeameroption{show notes}
%\setbeameroption{hide notes}
%\setbeameroption{show only notes}

%\setbeamercovered{transparent}
%\setbeamertemplate{navigation symbols}{}

\usepackage[ngerman]{babel}

\usepackage[utf8]{inputenc}

\usepackage{csquotes}
\usetheme{Darmstadt}

\usepackage{graphicx}
\usepackage{subcaption}
\usepackage{hyperref}


\author{nanooq}
\title{G20 und GG §20 Widerstandsrecht}
\subtitle{Wehrhafte Demokratie gegen wehrhafte Demokraten: \\ Rechte, Grenzen, Grauzonen der Ereignisse um den G20}
%\logo{}
\institute{Hackspace Siegen e.~V.\\
https://github.com/nanooq/mrmcd-2017-g20-gg20\\
https://cfp.mrmcd.net/2017/talk/ZZEFRH/}
\date{2017-09-01 22:00 \\  Prachtgarten @ MRMCD2017}
\subject{G20 und GG §20 Widerstandsrecht}

\begin{document}
	\frame[plain]{\maketitle}
	
	\begin{frame}
		\frametitle{Inhalt}
		\tableofcontents
		
	\end{frame}

	\begin{frame}
	\frametitle{Teil 1: UN und G20} 
	\section{Teil 1: UN und G20}
	Was kann der G20, was die UN nicht kann? \par
	Was ist der G20?
	\vfill
	Es folgt eine kleine Gipfelkunde
	\end{frame}
 
 	\begin{frame}
 		\subsection{G20}
 	\frametitle{Gipfelkunde}  \framesubtitle{Herrschertreffen}
 	Solche Treffen in der Vormorderne werden als Herrschertreffen bezeichnet \dots
 	\note[item]{Das Herrschertreffen oder die Herrscherbegegnung ist ein Fachbegriff der Geschichtswissenschaft und bezeichnet persönliche Zusammentreffen von Monarchen als Mittel der Politik. Für die Treffen von Staats- und Regierungschefs hat sich im 20. Jahrhundert der Begriff Gipfeltreffen etabliert \cite{wiki:herrschertreffen}.}
	 \end{frame}
 
 	\begin{frame}
	\frametitle{Gipfelkunde}  \framesubtitle{Name}
	\dots aktuell steht \enquote{G20} für \enquote{Gruppe der zwanzig wichtigsten Industrie- und Schwellenländer}
	\end{frame}
 
  	\begin{frame}
 	\frametitle{Gipfelkunde}  \framesubtitle{Fokus}
 	\begin{itemize}
 		\item Wirtschaft \& Finanzen
 		\item Klima \& Migration
 		\item Frauenrechte \& Bildungschancen
 		\item Terrorismus
 	\end{itemize}
 	\note[item]{
 		\begin{itemize}
 			\item Was fällt eigentlich nicht unter diese Themen? 
 			\item Der Gipfel dauert 2 bis 3 Tage.
 			\item In denen werden diese globalen Probleme besprochen. Wurden sie jemals gelöst? 
 			\item Finanzen 1: 1997 gründete sich G20 und war bis 2008 als Finanzministertreffen - 2007 war Finanzkrise \cite{Arzt2017}
 			\item Finanzen 2: Nach der Größe der Volkswirtschaften, gemessen am BIP, würde die Schweiz zu den G20 gehören, ist doch das BIP der Schweiz größer als dasjenige von Südafrika, Saudi-Arabien und Argentinien. Nun Einladung zu Vorbereitungsarbeiten zum Gipfel. Über eine allfällige Einladung entscheidet jeweils das Vorsitzland der G20.
 			\item Zu Frauenrechte wurde 2014 die W20 (Women) gegründet
 			\item Der Krieg gegen den Terror und Terror stiegen seitdem
 			\item Klimawandel? Welcher Klimawandel?
 			\item Migrationsländer unterrepräsentiert
 		\end{itemize}
 	}
	 \end{frame}
 
 	\begin{frame}
	 	\frametitle{Gipfelkunde}  \framesubtitle{Weltkarte}
		\begin{figure}[h!]
			\includegraphics[width=0.8\textwidth]{images/g20}
			\caption{G20}
		\end{figure}
	 \end{frame}

	\begin{frame}
		\frametitle{Gipfelkunde}  \framesubtitle{Eckdaten}
		\begin{itemize}
			\item dauert 2 bis 3 Tage
			\item Abschlusserklärung
			\item im Vorfeld ausgehandelt von Unterhändlern \enquote{Sherpas}
			\item reichlich Fototermine, 
			\item Theaterbesuch oder Konzert
		\end{itemize}
	\end{frame}

	\begin{frame}
	\frametitle{Gipfelkunde}  \framesubtitle{Fototermine und Fotos}
	\begin{figure}[h!]
		\renewcommand{\figurename}{Foto} 
		\includegraphics[width=0.8\textwidth]{images/fotos_01_BundesregierungBergmann}
		\caption{G20 (2017), Bundesregierung / Bergmann}
	\end{figure}
	\end{frame}

	\begin{frame}
	\frametitle{Gipfelkunde}  \framesubtitle{Fototermine und Fotos}
	\begin{figure}[h!]
		\renewcommand{\figurename}{Foto} 
		\includegraphics[width=0.8\textwidth]{images/fotos_02_BundesregierungKugler}
		\caption{G20 (2017), Bundesregierung / Kugler}
	\end{figure}
	\end{frame}

	\begin{frame}
	\frametitle{Gipfelkunde}  \framesubtitle{Fototermine und Fotos}
	\begin{figure}[h!]
		\renewcommand{\figurename}{Foto} 
		\includegraphics[width=0.8\textwidth]{images/fotos_03_BundesregierungKugler}
		\caption{G20 (2017), Bundesregierung / Kugler}
	\end{figure}
	\end{frame}

	\begin{frame}
	\frametitle{Gipfelkunde}  \framesubtitle{Fototermine und Fotos}	
	\begin{figure}[h!]
		\renewcommand{\figurename}{Foto} 
		\includegraphics[width=0.8\textwidth]{images/fotos_04_BundesregierungBergmann}
		\caption{G20 (2017), Bundesregierung / Bergmann}
	\end{figure}
	\end{frame}

	\begin{frame}
	\frametitle{Gipfelkunde}  \framesubtitle{Fototermine und Fotos}
	\begin{figure}[h!]
		\renewcommand{\figurename}{Foto} 
		\includegraphics[width=0.8\textwidth]{images/fotos_05_BundesregierungGuengoer}
		\caption{G20 (2017), Bundesregierung / Güngör}
	\end{figure}
	\end{frame}

	\begin{frame}
	\frametitle{Gipfelkunde}  \framesubtitle{Fototermine und Fotos}
	\begin{figure}[h!]
		\renewcommand{\figurename}{Foto} 
		\includegraphics[width=0.8\textwidth]{images/fotos_06_BundesregierungKugler}
		\caption{G20 (2017), Bundesregierung / Kugler}
	\end{figure}
	\end{frame}

	\begin{frame}
	\frametitle{Gipfelkunde}  \framesubtitle{Fototermine und Fotos}
	\begin{figure}[h!]
		\renewcommand{\figurename}{Foto} 
		\includegraphics[width=0.8\textwidth]{images/fotos_07_BundesregierungBergmann}
		\caption{G20 (2017), Bundesregierung / Bergmann}
	\end{figure}
	\end{frame}

	\begin{frame}
	\frametitle{Gipfelkunde}  \framesubtitle{Fototermine und Fotos}
	\begin{figure}[h!]
		\renewcommand{\figurename}{Foto} 
		\includegraphics[width=0.8\textwidth]{images/fotos_08_BundesregierungGuengoer}
		\caption{G20 (2017), Bundesregierung / Güngör}
	\end{figure}
	\end{frame}

	\begin{frame}
	\frametitle{Gipfelkunde}  \framesubtitle{Fototermine und Fotos}
	\begin{figure}[h!]
		\renewcommand{\figurename}{Foto} 
		\includegraphics[width=0.8\textwidth]{images/fotos_09_BundesregierungKugler}
		\caption{G20 (2017), Bundesregierung / Kugler}
	\end{figure}
	\end{frame}

	\begin{frame}
		\frametitle{Gipfelkunde}  \framesubtitle{Ergebnisse}
		Was hat der G20 jemals für uns getan?
		\begin{figure}[h!]
			\renewcommand{\figurename}{Foto} 
			\includegraphics[width=0.5\textwidth]{images/was-hat-die-g20-je-fuer-uns-getan.jpg}
			\caption{Life of Brian(1979)}
		\end{figure}
	\end{frame}

	\begin{frame}
	\frametitle{Gipfelkunde}  \framesubtitle{Ergebnisse}
	Was hat der G20 jemals für uns getan?
	\begin{itemize}
		\item Aquädukte?
		\item Medizin?
		\item Erziehung?
		\item öffentlicher Ordnung?
		\item Bewässerung?
		\item Straßen?
		\item Volksgesundheit?
		\item Frieden?
	\end{itemize}
	\note[item]{
		\begin{itemize}
			\item Erziehung? Durch die stenge Hand
			\item öffentliche Ordnung? 
			\item Frieden?
		\end{itemize}
	}
	\end{frame}

	\begin{frame}
	\frametitle{Gipfelkunde}  \framesubtitle{Ergebnisse}
	Was hat der G20 jemals für uns getan?
	\note[item]{Nichts hat der G20 getan. Es gibt kein verbindlichen Beschluss über gar nichts. Nicht einmal Visionen.}
	\end{frame}

	\begin{frame}
		\frametitle{Gipfelkunde}  \framesubtitle{Vergangene Gipfel}
		\begin{itemize}
			\item 2016 G20 in Hangzhou (VRC): 170.000 Personen werden überprüft, 255 Fabriken in Zwangsschließung, Deutschen Welle nicht akkreditiert)
			\item 2008 G8 in Heiligendamm: Durchsuchungen vor dem Gipfeltreffen, 50.000 Gegendemonstranten allein auf der Auftaktdemonstration in Rostock, Bundeswehreinsatz, Satellitenaufklärung
			\item 2007 G7 auf Schloss Elmau: Landwirte dürfen ihre Felder nicht an Protestierende vermieten, 200 Haftcontainer bereitgestellt, 17 Staatsanwälte und 100 Richter, Betretungsverbote gegen Journalisten.
			\item 2001 G8 in Genua (Italien): Neuer Grundsatz: abgelegenen Ort wählen,Carlo Giuliani
			\item 1992 G7 in München: \enquote{bayerische Art} des \enquote{etwas härter Hinlangens} und den Münchner Kessel. 
		\end{itemize}
	\note[item]{
	\begin{itemize}
		\item \cite{Sieren2016} \cite{FAZ.net2016}
		\item 2007: \enquote{Doch die zentrale Auftaktdemonstration in Rostock kippt in gewalttätige Randale. Zu Beginn des Gipfeltreffens bringen G-8-Gegner mit Blockaden zudem den politischen Ablauf des Gipfels durcheinander. An den folgenden Tagen ist der Tagungsort in Heiligendamm zeitweise nicht mehr zu erreichen. Chaos an der Ostsee. Polizisten sind 33 Stunden am Stück im Einsatz und werden in der prallen Sonne oft stundenlang nicht mit Essen und Trinken versorgt. Demonstranten sperrt man in Käfige, einen Anwalt dürfen sie nicht anrufen.} \cite{Wadewitz2009} 
		\item 2008: \enquote{Und dann kam der Regen. Mitten in den großen Aufmarsch am späten Samstagnachmittag hinein öffnete sich der Himmel und Wassermassen gingen auf die Demonstranten hinab – vor allem auf ihr Camp. Die wenigen, die allen Widrigkeiten zum Trotz nach Oberbayern gekommen waren, standen vor einem Schlammbecken, Schlafsäcke, Decken und Rucksäcke waren durchnässt. Viele reisten noch in der Nacht ab. "Das muss man erst hinkriegen: Gestern die Demonstranten wegschwemmen und heute so ein Wetter", ätzte Alexander Dobrindt am nächsten Tag. Und so blieben letztlich so wenige G-7-Gegner übrig, dass selbst viele Polizisten an dem Aufwand zweifelten, den das Land Bayern betrieben hatte. Als am späten Sonntagvormittag eine Gruppe selbst ernannter Wanderer aus dem Camp es doch noch geschafft hatte, die B2 oberhalb von Garmisch-Partenkirchen zu besetzen, konnte man hören, wie ein junger Polizist aus Bayern angesichts der in vollkommen unnötiger Anzahl anwesenden Polizei ein "Das ist so sinnlos" in seinen roten Dreitagebart hineinbrummelte.} \cite{Erk2015}
		\item \enquote{Am Mittag des 20. Juli eskalierte die Situation in Genua. Der Zug der Tute Bianche und anderer linker Gruppen wurde von der Polizei mit Tränengas attackiert. Viele der 20.000 in einer schmalen Straße eingeschlossenen Menschen versuchten zu flüchten, zahlreiche andere antworteten auf die Angriffe der Carabinieri mit Steinwürfen. Auf der Via Montevideo und der Via Tolemaide wurden Autos angezündet, am Corso Torino brannte ein Einsatzfahrzeug der Carabinieri aus. Bei den Auseinandersetzungen in den Seitenstraßen wurde nahe der Piazza Alimonda der 23-jährige Carlo Giuliani von dem 20-jährigen Carabiniere Mario Placanica durch einen Kopfschuss getötet und von Filippo Cavataio, der am Steuer des Polizeiwagens saß, zweimal überrollt. Giuliani soll sich zuvor mit einem Feuerlöscher auf die Heckscheibe des Carabinierifahrzeuges zubewegt haben. Von zwei abgegebenen Schüssen traf eine Kugel Giuliani in den Kopf.Die Polizei gab später zu, während der Auseinandersetzungen weitere 15 Schüsse abgegeben zu haben.} \cite{wiki:g8genua2001} \cite{Matteoni2011}
		\item 1997 \cite{Kubitza2015}
	\end{itemize}}
	\end{frame}

	\begin{frame}
	\frametitle{Gipfelkunde}  \framesubtitle{Ergebnisse}
	Was hat der G20 jemals für uns getan?
	\note[item]{Nichts hat der G20 getan. Es gibt kein verbindlichen Beschluss über gar nichts. Nicht einmal Visionen.}
\end{frame}

	\begin{frame}
	\frametitle{Gipfelkunde}  \framesubtitle{Vergangene Gipfel: Lektion}
	\textbf{Vergangene Gipfel: Mögliche Lektionen}
		\begin{itemize}
			\item Deeskalation
			\item 2001 G8 in Genua (Italien): Neuer Grundsatz: abgelegenen Ort wählen
		\end{itemize}
	\note[item]{
	\enquote{Seit dem Gipfel und verstärkt nach den Terroranschlägen am 11. September 2001 gilt nach dem Summit policing der Grundsatz, für G8/G20-Gipfel einen Ort zu wählen, der möglichst abgelegen ist und gut abgesichert werden kann. Laut Tony Blair soll verhindert werden, dass die publizistische Wirkung von Protesten den Gipfel in den Augen der Öffentlichkeit ruiniert.}\cite{wiki:g8genua2001}\cite{Blair2010} }
	\end{frame}

	\begin{frame}
	\frametitle{Gipfelkunde}  \framesubtitle{Verwechslungsgefahr}
	\textbf{G20 (Schwellenländer)}
	\begin{figure}[h!]
%		\renewcommand{\figurename}{Foto} 
		\includegraphics[width=0.7\textwidth]{images/verwechslung-g20}
		\caption{G20 (Entwicklungsländer)}
	\end{figure}
	\begin{itemize}
		\item Themen: Agrarpolitik.
		\item Forderung 1: Abbau Agrarsubventionen 
		\item Forderung 2: Abbau Importbeschränkungen
		\item Ergebnis: Gescheitert.
	\end{itemize}
	\note[item]{
		\begin{itemize}
			\item Abbau in USA und EU
			\item Da die Vorschläge der USA und der EU nicht den Ergebnissen der Doha-Runde entsprachen und die G20-Länder geschlossen bei ihren Forderungen blieben, kam es zum Scheitern der Verhandlungen.\cite{wiki:g20schwellen} 
			\item TTIP hier, bitte?
		\end{itemize}
	}
	\end{frame}

	 \begin{frame}
	\frametitle{UN} 
	\subsection{UN}
	\textbf{United Nations}
	\begin{itemize}
		\item 193 Staaten zusammen geschlossen
		\item Charta mit Aufgaben
		\begin{itemize}
			\item Wirtschafts- und Sozialrat (Art. 61–72)
			\item Regionale Abmachungen (Art. 52–54)
		\end{itemize}
		\item Fertige Infrastruktur
		\item Beschlussfähigkeit
	\end{itemize}
	
	\note[item]{
		\begin{itemize}
			\item 193 Staaten sind 9,65 G20s
			\item Kapitel IX: Internationale Zusammenarbeit auf wirtschaftlichem und sozialem Gebiet (Art. 55–60)
			\item Kapitel VIII: Regionale Abmachungen (Art. 52–54)
		\end{itemize}
	}
\end{frame}

	 \begin{frame}
	\frametitle{UN und G20} 
	\subsection{Gegenüberstellung}
	Wo globale Herausforderungen besprechen und Lösungen angehen?
	\begin{figure}[h!]
		\includegraphics[width=0.5\textwidth]{images/un}
		\includegraphics[width=0.5\textwidth]{images/g20}
	\end{figure}
	Es geht um Wirtschaft, Finanzen, Klima, Migration, Frauenrechte, Bildungschance, Terrorismus\par
	\note[item]{
		\begin{itemize}
			\item Die G20-Themen sind auch in den UN-Themen enthalten, wobei \dots
			\item \dots ich regel sowas alles bevorzugt mit meiner Familie, meinen Freunden, meinen Mitbewohnern und meinen Nachbarn.
			\item  Ich habe das Gefühl, in den letzten 15 Jahren hätte die UN irgendwie an Aufmerksamkeit verloren. Wie die Sci-Fi Welt von Star-Trek mit der friedlichen Weltregierung. Aber diesen Gedanken werde ich ein anderes Mal verfolgen.
		\end{itemize}
	}
\end{frame}

	\begin{frame}
		\frametitle{Teil 2: Rechte und Strategien}
		\section{Teil 2: Rechte und Strategien}
		\begin{itemize}
			\item Naturgesetz
			\item Grundgesetz
			\item Polizeistrategien
		\end{itemize}
	\end{frame}

	\begin{frame}
	\frametitle{Naturgesetze als Recht?}
	\subsection{Rechte und Gesetze}
	Naturgesetz: $ F = G \dfrac{m_1 m_2}{r^{2}} $   
	\note[item]{
		\begin{itemize}
			\item Das Gravitationsgesetz von Isaac Newton lautet \enquote{Die Anziehungskraft F zwischen zwei Massen m 1 und m 2 ist proportional der Größe der Massen und umgekehrt proportional zum Abstandquadrat r 2. G ist dabei ein Proportionalitätsfaktor, der die Massen m 1 und m 2 und das Inverse des Abstandsquadrats 1 durch r 2 miteinander in Relation setzt.}
			\end{itemize}
			\item Wenn ich die Anziehungskraft zwischen meiner Freundin und mir erhöhen möchte, dann kann ich  dicker werden und / oder näher zu ihr hin. 
			\item \enquote{Es gibt keine technische Lösung für soziale Probleme}
		}
	\end{frame}

	\begin{frame}
	\frametitle{Grundgesetz für die Bundesrepublik Deutschland Art 1 }
	(1) Die Würde des Menschen ist unantastbar. Sie zu achten und zu schützen ist Verpflichtung aller staatlichen Gewalt.\par
	(2) Das Deutsche Volk bekennt sich darum zu unverletzlichen und unveräußerlichen Menschenrechten als Grundlage jeder menschlichen Gemeinschaft, des Friedens und der Gerechtigkeit in der Welt.\par
	(3) Die nachfolgenden Grundrechte binden Gesetzgebung, vollziehende Gewalt und Rechtsprechung als unmittelbar geltendes Recht.
	\note[item] {
		\begin{itemize}
		\item Es ist am 23.05.1949 erlassen worden, vierzig Jahre vor dem mysteriösen Tod unseres CCC-Mitglied Celine Hagbard.
		\item Absatz 1
		\begin{itemize}
			\item Naturrecht: Menschen entdecken das Recht. Weil die Würde des Menschen unantastbar und es \enquote{Verpflichtung aller staatlichen Gewalt} ist, \enquote{sie zu achten und zu schützen}, darum bekennt sich das deutsche Volk zu Menschenrechten.
			\item Posivrecht: Menschen erschaffen das Recht.
			\item Die Unantastbarkeit der Würde des Menschen ist jedenfalls kein Naturgesetz. 
		\end{itemize}
		\item Absatz 2
		\begin{itemize}
			\item Der NS-Staat vor der BRD hat diesen Artikel gebrochen. Viele Staaten davor und danach, auch die Bundesrepublik selber.
			\item Zu den Brüchen durch die BRD gehören: Embryonenschutz\cite{Boeckenfoerde2003}\cite{Boeckenfoerde2004}\cite{Nettesheim2005}, Folterverbot\cite{Bielefeldt2004}, Abschiebungen, Lauschangriffe , Strafverfahren\cite{Gusy2005} 
			\item Auch dieser Absatz ist eine Reaktion auf menschenverachtenden Ereignisse des Zweiten Weltkrieges. Aus fundamentalen, ethischen und moralischen Gedanken, soll sich das nicht wiederholen, also werden die Menschenrechte integriert.
		\end{itemize}
		\item Absatz 3
			\begin{itemize}
				\item Feature zur Weimarer Verfassung. Die Gesetzgebung, die vollziehende Gewalt und die Rechtsprechung wird an dieses als unmittelbar geltendes Recht gebunden. 
				\item Damit werden die Grundrechte (Art. 1 bis 10) zu Rechtsansprüchen des Einzelnen gegenüber dem Staat. 
				\item Eingriffe in Grundrechte durch etwas anderes als Grundrechte sind unzulässig. Und das berührt die Fragestellung in diesem Vortrag: Versammlungsrecht oder Strafverfahren? 
				\item Der Bürger kann unter Berufung auf die Grundrechte klagen. Sollte der Bürger nach Erschöpfung des Rechtswegs der Meinung sein, dass immer noch eine Grundrechtsverletzung besteht, kann er das Bundesverfassungsgericht im Wege einer Verfassungsbeschwerde anrufen.	
			\end{itemize}
		\end{itemize}
		}
	\end{frame}

	\begin{frame}
	\frametitle{Grundgesetz für die Bundesrepublik Deutschland Art 79 \enquote{Ewigkeitsklausel} }
	Absatz 3\par
	\enquote{Eine Änderung dieses Grundgesetzes, durch welche die [\dots] in den Artikeln 1 und 20 niedergelegten Grundsätze berührt werden, ist unzulässig.}
	\vfill
	Daraus folgt: Wem das Grundgesetz nicht gefällt, 
	\begin{itemize}
		\item der kann auswandern.
		\item der kriegt auf's Maul? (Artikel 20 GG später)
	\end{itemize}
	\end{frame}

	\begin{frame}
	\frametitle{Grundgesetz für die Bundesrepublik Deutschland Art 8 \enquote{Versammlungsfreiheit} (1)}
	(1) Alle Deutschen haben das Recht, sich ohne Anmeldung oder Erlaubnis friedlich und ohne Waffen zu versammeln.\par
	(2) Für Versammlungen unter freiem Himmel kann dieses Recht durch Gesetz oder auf Grund eines Gesetzes beschränkt werden.	
	\note[item]{
	\begin{itemize}
		\item Ursprug ist § 161 der Paulskirchenverfassung von 1848. 
		\item Damals gab es staatlicher Versuche, Versammlungen einzuschränken. Zum Beispiel 1819 die Karlsbader Beschlüsse und nach 1832 Repressionengegen Teilnehmer des Hambacher Fests. 
		\item Die Weimarer Republik kannte mit Art. 123 eine Versammlungsfreiheit gewährt, die per Verordnung des Reichspräsidenten zum Schutz von Volk und Staat außer Kraft gesetzt wurde.		
		\item Nach Föderalismusreform 2006 verfassen einigen Bundesländer eigene Versammlungsgesetze. Die von Bayern wird gerade verfassungsrechtlich überprüft.
	\end{itemize}}
	\end{frame}

	\begin{frame}
	\frametitle{Grundgesetz für die Bundesrepublik Deutschland Art 8 \enquote{Versammlungsfreiheit} (2)}
	\begin{itemize}
		\item \enquote{But I always say, one's company, two's a crowd, and three's a party} (Andy Warhol)
		\item Ohne Anmeldung?
		\item Friedlich und ohne Waffen?
		\item Unter freiem Himmel?
		\item \enquote{Alles weitere regelt ein Bundesgesetz}
	\end{itemize}
	\note[item]{
	\begin{itemize}	
		\item Es ist noch offen, wieviele Personen man braucht um einer Versammlung darzustellen. Eher ist man eine Gesellschaft bürgerlichen Rechts.
		\item Nach Föderalismusreform 2006 verfassen einigen Bundesländer eigene Versammlungsgesetze. Die von Bayern wird gerade verfassungsrechtlich überprüft.
		\item In den Versammlungsgesetzen werden auch Anmelde- und Genehmigungspflichten bestimmt.
		\item Mit \enquote{freier Himmel} ist der öffentliche Raum gemeint.
	\end{itemize}}
	\end{frame}

	\begin{frame}
	\frametitle{Versammlungsrecht gilt für \dots (1)}
	örtliche Zusammenkünfte mehrerer Personen zur gemeinschaftlichen, auf die Teilhabe an der öffentlichen Meinungsbildung gerichteten Erörterung oder Kundgebung
	\begin{figure}[h!]
	\renewcommand{\figurename}{Foto} 
	\includegraphics[width=0.8\textwidth]{images/nicht-fuer-friedensdemo}
	\caption{{\small epd-bild/Hartwig Lohmeyer: Demo im Oktober 1981 in Bonn gegen die Stationierung von Pershing 2 Raketen in der BRD.}}
	\end{figure}	
	\end{frame}


	\begin{frame}
	\frametitle{Versammlungsrecht gilt für \dots (2)}
	friedlichen Gruppen.
	\vfill
	Trotz unfriedlicher Gruppen, gilt für die friedlichen Gruppen weiterhin uneingeschränkt das Grundrecht der Versammlungsfreiheit.	
	\end{frame}


	\begin{frame}
	\frametitle{Versammlungsrecht gilt nicht für \dots (1)}
	\begin{figure}[h!]
		\renewcommand{\figurename}{Foto} 
		\includegraphics[width=0.8\textwidth]{images/nicht-fuer-loveparade}
		\caption{dpa, Ausgelassen feiern Raver auf der Straße des 17. Juni in Berlin die 13. Love Parade am 21.07.2001\cite{Online2014}}
		Der Massenparty fehlt ein politische Meinungsbild.
	\end{figure}
	\note[item]{
		Ein gemeinsamer, meinungsbildener Zweck ist umstritten.
	}
	\end{frame}

\begin{frame}
	\frametitle{Versammlungsrecht gilt nicht für \dots (2)}
	\begin{figure}[h!]
		\renewcommand{\figurename}{Foto} 
		\includegraphics[width=0.8\textwidth]{images/nicht-fuer-gaffer}
		\caption{dpa, Drei Männer behindern am 5. Juli 2015 in Bremervörde nach einem tödlichen Unfall die Arbeit der Rettungskräfte.}
		Fehlender gemeinsamer Zweck.
	\end{figure}
\end{frame}

\begin{frame}
	\frametitle{Versammlungsrecht gilt nicht für \dots (3)}
	\begin{figure}[h!]
		\renewcommand{\figurename}{Foto} 
		\includegraphics[width=0.7\textwidth]{images/nicht-fuer-leeroy}
		\caption{Internet, World of Warcraft, Battlevorbereitung}
		Virtuellen Treffen fehlt die Örtlichkeit. \enquote{At least, I have chicken.}
	\end{figure}
	\note[item] {
		Nicht erforderlich ist, dass die Versammlung ortsfest ist. Daher werden auch Demonstrationszüge von der Versammlungsfreiheit geschützt.
	}
\end{frame}

\begin{frame}
	\frametitle{Versammlungsrecht gilt nicht für \dots (4)}
	\begin{figure}[h!]
		\renewcommand{\figurename}{Foto} 
		\includegraphics[width=0.7\textwidth]{images/nicht-fuer-gewalttaeter}
		\caption{Twitter / @verbraak, Gewalt bei einer Hooligan-Demo in Köln am 27.10.2014}
	\end{figure}
	Für feindselige, aufrührerische und bewaffnete Zusammenschlüsse gibt es keinen Grundrechtsschutz.
	\note[item] {
		Nicht erforderlich ist, dass die Versammlung ortsfest ist. Daher werden auch Demonstrationszüge von der Versammlungsfreiheit geschützt.
	}
	\end{frame}

	\begin{frame}
	\frametitle{Grundgesetz für die Bundesrepublik Deutschland Art 20}
	(1) Die Bundesrepublik Deutschland ist ein demokratischer und sozialer Bundesstaat.\par
	(2) Alle Staatsgewalt geht vom Volke aus. Sie wird vom Volke in Wahlen und Abstimmungen und durch besondere Organe der Gesetzgebung, der vollziehenden Gewalt und der Rechtsprechung ausgeübt.\par
	(3) Die Gesetzgebung ist an die verfassungsmäßige Ordnung, die vollziehende Gewalt und die Rechtsprechung sind an Gesetz und Recht gebunden.\par	
	(4) Gegen jeden, der es unternimmt, diese Ordnung zu beseitigen, haben alle Deutschen das Recht zum Widerstand, wenn andere Abhilfe nicht möglich ist.
	\note[item]{
		\begin{itemize}
			\item Absatz 1: Unsere Staatsbestimmungen sind Demokratie, Sozialstaat, Föderalismus und die Republik. Republik im Sinne von Nicht-Monarchie.
			\item Absatz 2: \enquote{Alle Gewalt geht vom Volke aus, weshalb es bekämpft werden muss}. Die drei Gewalten und ihre Teilung werden hier ebenfalls festgestellt. Die Polizei gehört zu Exekutive.  \enquote{Alle} bedeutet, dass es keine Gewalt geben darf, die nicht vom Volk begründet ist. 
			\item Absatz 3: Hier wird den drei Gewalten Rechtsstaatlichkeit vorgeschrieben.
			\item Absatz 4: Das Widerstandsrecht ist 1968 mit der Notstandsgesetzgebung eingeführt worden, vorher gab es weder Notstandsgesetze noch Widerstandsrecht, aber war es notwendig? Die Verhätnismäßigkeit muss gewahrt bleiben.
		\end{itemize}
	}
	\end{frame}

	\begin{frame}
	\subsection{Die Polizei, dein Freund und Helfer}
	\frametitle{Die Polizei, dein Freund und Helfer (1)}
	\begin{itemize}
		\item Exekutivorgan
		\item Öffentliche Sicherheit und Ordnung (Polizeirecht)
		\item Notfallhilfe
		\item Strafverfolgung
		\item Gefahrenabwehr (Polizeirecht)
		\item Unmittelbarer Zwang
		\item A.C.A.B: All Cops are (Vollzugs)Beamte
	\end{itemize}
	\note[item]{
		\begin{itemize}
			\item Die Exekutive besteht aus Regierung, Verwaltung und vielen nach gelagerten Organe.
			\item Beim Einschreiten darf die Polizei die Gewalt anwenden. Natürlich nur verhältnismäßig und in gesetzlichen Grenzen.
			\item \enquote{Bulle} kommt wohl vom niederländischen \enquote{bol}, was \enquote{Kopf}, oder \enquote{Kluger Mensch heißt}.
		\end{itemize}
	}
	\end{frame}

	\begin{frame}
	\frametitle{Die Polizei, dein Freund und Helfer (2)}
	Es gibt Polizei in Wohlfahrtstaaten,\par
	Polizeistaaten\par
	und Nachtwächterstaaten.
	\end{frame}

	\begin{frame}
	\frametitle{Einsatzmittel der Polizei}
	Für Polizisten gilt weder das 
	\begin{itemize}
		\item Vermummungsverbot noch
		\item das Waffenverbot und sie haben
		\item einen Haufen Einsatzmittel
	\end{itemize}
	\end{frame}



\begin{frame}
		\frametitle{Einsatzmittel der Polizei, Allgemeine Mittel }
		\begin{itemize}
			\item Werkzeuge
			\item Lichttechnik
			\item Kartenmaterial 
			\item Löschmittel
			\item Leitstellen
			\item BOS-Funksystem, Funkgeräte, Funkmelde- und -leitsysteme, 
			\item Gebrauchshunde 
			\item Atemschutzmaske 
			\item Schutzhelme
			\item Warnleuchten
			\item Leitkegel
			\item Absperrbänder
			\item Notmeldeeinrichtungen
			\item Mehrzweckzüge
			\item Dosimeter sowie Dosisleistungsmessgeräte.
		\end{itemize}
	\note[item] {
		\begin{itemize}
			\item Der BOS-Funk ist ein sogenannter \enquote{nömL} (nichtöffentlicher mobiler UKW-Landfunkdienst. Steht für Behörden und Organisationen mit Sicherheitsaufgaben.
			\item 
		\end{itemize}
	}
	\end{frame}
	
	\begin{frame}
		\frametitle{Einsatzmittel der Polizei, Spezielle Mittel }
		\begin{itemize}
			\item Polizeifahrzeuge (wie Streifenwagen und Wasserwerfer)
			\item Polizeifunk
			\item Handschellen
			\item Waffen und Munition
			\item Schlagstöcke
			\item Pfefferspray
			\item Körperschutzausstattungen und Schutzwesten
			\item Polizeipferde und -hunde
			\item Radar
			\item Schutzschilde
			\item Megafone
			\item Rammen
			\item Nagelgurte 
			\item Sperrgitter
		\end{itemize}	
	\end{frame}

	\begin{frame}
	\frametitle{Einsatzmittel der Polizei, Spezielle Mittel (1) }
	\begin{figure}[h!]
		\renewcommand{\figurename}{Foto} 
		\includegraphics[width=0.7\textwidth]{images/spezielle-mittel-waffen1}
		\caption{Deutsche Polizisten mit Maschinenpistolen}
	\end{figure}
\end{frame}	

	\begin{frame}
	\frametitle{Einsatzmittel der Polizei, Spezielle Mittel (2) }
	\begin{figure}[h!]
		\renewcommand{\figurename}{Foto} 
		\includegraphics[width=0.7\textwidth]{images/spezielle-mittel-polizeiausruestung2.jpg}
		\caption{Der neue Polizei-Panzerwagen \enquote{Survivor}}
	\end{figure}
	\end{frame}

	\begin{frame}
	\frametitle{Einsatzmittel der Polizei, Spezielle Mittel (3) }
	\begin{figure}[h!]
		\renewcommand{\figurename}{Foto} 
		\includegraphics[width=0.7\textwidth]{images/spezielle-mittel-polizeifahrzeuge1}
		\caption{Verschiedene schwere Einsatzfahrzeuge der deutschen Polizei bei einer Demo}
	\end{figure}
	\end{frame}

	\begin{frame}
	\frametitle{Einsatzmittel der Polizei, Spezielle Mittel (4)}
	\begin{figure}[h!]
		\renewcommand{\figurename}{Foto} 
		\includegraphics[width=0.7\textwidth]{images/spezielle-mittel-polizeiausruestung1.jpg}
		\caption{Der neue Polizei-Panzerwagen \enquote{Survivor}}
	\end{figure}
	\note[item]{
		Euro III Klasse
	}
	\end{frame}

	\begin{frame}
	\frametitle{Einsatzmittel der Polizei, Spezielle Mittel (5)}
	\begin{figure}[h!]
		\renewcommand{\figurename}{Foto} 
		\includegraphics[width=0.7\textwidth]{images/spezielle-mittel-polizeifahrzeuge2.jpg}
		\caption{Polizeisonderfahrzeug SW 4, Hersteller Thyssen und DB AG}
	\end{figure}
	\end{frame}

	\begin{frame}
	\frametitle{Weiteres zur Polizei}
	\begin{itemize}
		\item Einkaserniert, stehendes Herr
		\item Ausbildung in Gewaltanwendung (unmittelbarer Zwang)
		\item Ausbildung in Deeskalation
		\item Medizinische Betreuung
	\end{itemize}
	\end{frame}


	\begin{frame}
	\frametitle{Fazit}
	Behauptung:\par 
	Eine ausgeschlafene, gut versorgte, gut ausgebildete und gut ausgerüstete Polizei ist in der Lage das Versammlungsrecht friedlicher Gruppen zu schützen.
	\vfill
	Eine Frage der Strategie?
	\end{frame}


	\begin{frame}
	\subsection{Polizeistrategien}
	\frametitle{Polizeistrategien}
	\begin{itemize}
		\item Das \enquote{Mildeste Mittel}
		\item Die \enquote{Harte Linie}
		\item Welche Strategie wurde in Hamburg gewählt?
	\end{itemize}
	\end{frame}

	\begin{frame}
	\frametitle{Polizeistrategie, Mildestes Mittel}
	\begin{itemize}
		\item verfassungsrechtlich das einzig gebotene Mittel
		\item Verhältnismäßigkeit von Gewaltandrohung und -anwendung 
		\item Gesetzestreue Opposition schützen ist Aufgabe des Staates
		\item Das \enquote{mildeste Mittel} ist
		\begin{itemize}
			\item das effizientere Mittel
			\item hat die höchste Legitimation
			\item isoliert die Gewalttäter
		\end{itemize} 
	\end{itemize}
\end{frame}

	\begin{frame}
	\frametitle{Polizeistrategie, Konflikteeskalation}
	\begin{figure}
		\renewcommand{\figurename}{Foto} 
		\includegraphics[width=\textwidth]{images/konflikteeskalation-nach-glasl}
		\caption{Wiki, Benutzer:Sampi, Modell der Konflikteskalation nach Friedrich Glasl}
	\end{figure}
	\note[item]{
		\begin{enumerate}
			\item Konflikte beginnen mit Spannungen. Der Konflikt könnte tiefere Ursachen haben.
			\item Debatte, Konfliktpartner überlegen sich Strategien, um den anderen von ihren Argumenten zu überzeugen. Minungsverschiedenheit, Streit, Unter-Druck-Setzen, Schwarz-Weiß-Denken.
			\item Taten statt Worte, Gespräche abbrechen, Das Mitgefühl für den \enquote{anderen} geht verloren.
			\item Koalitionen mit Sympathisanten für seine Sache suchen. Gegner denunzieren, Ziel ist nicht die Sache, sondern den Konflikt zu gewinnen.
			\item Gesichtsverlust, Gegner soll in seiner Identität vernichtet werden durch alle möglichen Unterstellungen. Vollständiger Vertrauensverlust. Verlust der moralischen Glaubwürdigkeit.
			\item Drohstrategien, Machtdarstellung, Sanktionspotenzial darstellen
			\item Begrenzte Vernichtung, Gegner mit allen Tricks empfindlich schaden. Der Gegner wird nicht mehr als Mensch wahrgenommen. Ab hier wird ein begrenzter eigener Schaden schon als Gewinn angesehen, sollte der des Gegners größer sein.
			\item Zersplitterung: Das Unterstützersystem des Gegners soll mit Vernichtungsaktionen zerstört werden.
			\item Gemeinsam in den Abgrund: Ab hier kalkuliert man die eigene Vernichtung mit ein, um den Gegner zu besiegen.
		\end{enumerate}
		}
	\end{frame}


	\begin{frame}
	\frametitle{Polizeistrategie, Deeskalation}
	Deeskalationsstufen
	\begin{enumerate}
		\item Verhinderung bzw. Verminderung aggressionsauslösender Reize
		\item Wahrnehmung, Interpretation und Bewertung von erregten Verhaltensweisen und deren Folgen
		\item Verständnis der Ursachen und Beweggründe aggressiver Verhaltensweisen
		\item Verbale Deeskalation in Eskalationssituationen
		\item Sicherheitshinweise und Abwehrtechniken bei Angriffen
	\end{enumerate}
	\note[item]{
	\begin{itemize}
		\item Vermeidung staatlicher Drohgebärden, indem die fremdgalaktisch gekleideten Kampftruppen unsichtbar in Seitenstraßen positioniert wurden,
		\item kein "automatisches" Reagieren auf Provokationen:"Die Macht muss sich die Entscheidung zum Handeln vorbehalten.
		\item Polizei in der Demo mitlaufen lassen
		\item keine hysterischen Lautsprecheransagen der Polizei-Einsatzleitung. 
		\item Einsatzleitung begibt sich zu den Demonstrationsverantwortlichen begibt und verabredet die Demonstrationsauflösung konkret und ruhig  anstatt sie gebieterisch zu verkünden.
		\item Keine Kessel, auch nicht um Entwicklung zur Unfriedlichkeit vorzubeugen. 
	\end{itemize}
}
	\end{frame}

	\begin{frame}
	\frametitle{Polizeistrategie, Harte Linie (1)}
	Leitender Polizeidirektor Hartmut Dudde wird häufig mit dem Begriff der \enquote{Hamburger Linie} in Verbindung gebracht, unter der ein hartes Vorgehen der Hamburger Polizei verstanden wird.\par 
	Mehrere Polizei-Einsätze, an denen er in Leitungsverantwortung beteiligt war, wurden im Nachgang gerichtlich für rechtswidrig erklärt. (\url{https://de.wikipedia.org/wiki/Hartmut_Dudde})
	\vfill
	Im folgenden \enquote{Harte Linie} genannt.
	
\end{frame}

	\begin{frame}
	\frametitle{Polizeistrategie, Harte Linie (2)}
	\begin{itemize}
		\item Drohungen
		\item Bruch von Vereinbarungen und Recht
		\item Unverhältnismäßig hohe Gewaltanwendung
	\end{itemize}
	\vfill
	Feuer und Flamme der Repression!
	\note[item]{
	Fronten, Martialisches Auftreten
	}
\end{frame}

	\begin{frame}
	\frametitle{Polizeistrategie, Harte Linie (3)}
	\begin{itemize}
		\item Die \enquote{Harte Linie} ist nicht verfassungsmäßig
		\item Die \enquote{Harte Linie} vertieft bestehende Widersprüche
		\item Die \enquote{Harte Linie} ist staatliche Repression
	\end{itemize}
	\vfill
	Feuer und Flamme der Repression!
\end{frame}

	\begin{frame}
	\section{Teil 3: Verlauf G20}
	\frametitle{Teil 3: Verlauf G20}
	\begin{enumerate}
		\item Verlauf der Polizeistrategie
		\item Verlauf friedlicher Demos
		\item Nachbereitung
	\end{enumerate}
\end{frame}

	\begin{frame}
	\subsection{Verlauf der Polizeistrategie}
	\frametitle{Verlauf der Polizeistrategie (1)}
	Drohungen:\par 
	\enquote{Sollte es zu einer erfolgreichen Blockade kommen, würden sich die Teilnehmer beim Aufeinandertreffen mit der Kolonne in Gefahr bringen.} Innensenator Grote (Stern, 10. Mai 2017 19:44)
	\note[item]{
	Von einem \enquote{Festival der Demokratie} sprach Hamburgs Innensenator Andy Grote vor dem G20-Gipfel}
\end{frame}

	\begin{frame}
	\frametitle{Verlauf der Polizeistrategie (2)}
	Bruch von Vereinbarungen:\par 
	\enquote{Weil der Versammlungsleiter der Autonomendemo nach eigener Darstellung mit dem Einsatzleiter der Polizei eine Übereinkunft erzielt hatte: Kapuzen und Sonnenbrillen würden geduldet, Schals und Tücher seien aber von der unteren Gesichtshälfte zu entfernen.} (Zeit 7. Juli 2017, 16:42 Uhr)
\end{frame}

	\begin{frame}
	\frametitle{Verlauf der Polizeistrategie (3)}
	Bruch von Recht:\par 
	\enquote{Die Eingriffe der Polizei am Sonntag sind ohne Rechtsgrundlage durchgeführt worden.} Verfassungsrechtler Ulrich Karpen (Morgenpost 03.07.17, 22:01 Uhr)
\end{frame}

	\begin{frame}
	\frametitle{Verlauf der Polizeistrategie (4)}
	Unverhältnismäßig hohe Gewaltanwendung:\par 
	\enquote{Unter den Bildern, die der Gipfel lieferte, sind auch jene, die den größten Tabubruch bezeugen: der Einsatz von Sondereinsatzkommandos gegen Demonstrant*innen – mit der Freigabe, im Zweifel zu schießen.} (taz 13.07.2017)
\end{frame}

	\begin{frame}
	\frametitle{Verlauf der Polizeistrategie (5)}
		In Hamburg brannten 25 Autos aus, ein Supermarkt wurde geplündert, Beamte und Demonstranten verletzt.
	\begin{figure}
		\renewcommand{\figurename}{Foto} 
		\includegraphics[width=0.8\textwidth]{images/polizeistrategie1.jpg}
		\caption{imago/Markus Heine}
	\end{figure}
\end{frame}

	\begin{frame}
	\subsection{Verlauf friedlicher Demos}
	\frametitle{Verlauf friedlicher Demos}
	\note[item]{
	Scherz.}
\end{frame}

	\begin{frame}
	\frametitle{Verlauf friedlicher Demos: G20 Protestwelle (1)}
		02.07.2017. Bündnis aus Naturfreunde, Greenpeace und Campact.\par
		Gegen Autokraten auf der Alster und ein Marsch.\par 
		Zehntausende Teilnehmer
	\begin{figure}
		\renewcommand{\figurename}{Foto} 
		\includegraphics[width=0.8\textwidth]{images/g20-protestwelle1}
		\caption{dpa}
	\end{figure}
	\note[item]{
		G20 Protestwelle ist ein Bündnis aus Naturfreunde, Greenpeace und Campact: Sie stellen sich gegen Autokraten und planen eine Demo auf der Alster mit Flößen, Kanus und Booten. Danach soll der Marsch in Richtung St. Pauli ziehen. Zehntausende Teilnehmer werden erwartet.
	}
\end{frame}

	\begin{frame}
	\frametitle{Verlauf friedlicher Demos: G20 Protestwelle (2)}
	\begin{figure}
		\renewcommand{\figurename}{Foto} 
		\includegraphics[width=0.8\textwidth]{images/g20-protestwelle2}
		\caption{dpa}
	\end{figure}
\end{frame}

	\begin{frame}
	\frametitle{Verlauf friedlicher Demos: G20 Protestwelle (3)}
	\begin{figure}
		\renewcommand{\figurename}{Foto} 
		\includegraphics[width=0.8\textwidth]{images/g20-protestwelle3}
		\caption{dpa/Axel Heimken}
	\end{figure}
\end{frame}

	\begin{frame}
	\frametitle{Verlauf friedlicher Demos: G20 Protestwelle (4)}
	\begin{figure}
		\renewcommand{\figurename}{Foto} 
		\includegraphics[width=0.8\textwidth]{images/g20-protestwelle4}
		\caption{dpa/Axel Heimken}
	\end{figure}
\end{frame}

	\begin{frame}
	\frametitle{Verlauf friedlicher Demos: Hard Cornern (1)}
		Abend 04.07.2017. Bündnis \enquote{Alles Allen}, Freies Sender Kombinat.\par
		Öffentliche Straßenparties. Ca. tausend Teilnehmer.\par
		Polizei räumt die Kreuzung Neuer Pferdemarkt mit Wasserwerfern.
	\begin{figure}		
		\renewcommand{\figurename}{Foto} 
		\includegraphics[width=0.8\textwidth]{images/hard-cornern1}
		\caption{picture-alliance / afp / Henrik Josef Boerge}
	\end{figure}
	\note[item]{
	Am Abend des 4. Juli veranstaltete das Bündnis \enquote{Alles Allen} und Freies Sender Kombinat das \enquote{hedonistische Massencornern} (öffentliche Straßenparties). Daran beteiligten sich vor allem in Nachbarvierteln zum Gipfelgelände Tausende, besetzten Bürgersteige und Straßenecken, errichteten Infostände und spielten Musik. Später räumte die Polizei die Kreuzung Neuer Pferdemarkt mit Wasserwerfern.
	}
\end{frame}

	\begin{frame}
	\frametitle{Verlauf friedlicher Demos: Lieber tanz ich als G20 (1)}
		05.07.2017. Nachttanzdemo an den Landungsbrücken\par
		und durch die Innenstadt.\par
		15.000 Teilnehmer
	\begin{figure}
		\renewcommand{\figurename}{Foto} 
		\includegraphics[width=0.8\textwidth]{images/lieber-tanz-ich-als-g20-1}
		\caption{}
	\end{figure}
\end{frame}

	\begin{frame}
	\frametitle{Verlauf friedlicher Demos: Lieber tanz ich als G20 (1)}
		05.07.2017. Nachttanzdemo an den Landungsbrücken\par
		und durch die Innenstadt.\par
		15.000 Teilnehmer
	\begin{figure}
		\renewcommand{\figurename}{Foto} 
		\includegraphics[width=0.8\textwidth]{images/lieber-tanz-ich-als-g20-1}
		\caption{}
	\end{figure}
\end{frame}

	\begin{frame}
	\frametitle{Verlauf friedlicher Demos: Lieber tanz ich als G20 (2)}
	\begin{figure}
		\renewcommand{\figurename}{Foto} 
		\includegraphics[width=0.8\textwidth]{images/lieber-tanz-ich-als-g20-2}
		\caption{}
	\end{figure}
\end{frame}

	\begin{frame}
	\frametitle{Verlauf friedlicher Demos: Lieber tanz ich als G20 (3)}
	\begin{figure}
		\renewcommand{\figurename}{Foto} 
		\includegraphics[width=0.8\textwidth]{images/lieber-tanz-ich-als-g20-3}
		\caption{Laura Lagershausen }
	\end{figure}
\end{frame}

	\begin{frame}
	\frametitle{Verlauf friedlicher Demos: Lieber tanz ich als G20 (4)}
	\begin{figure}
		\renewcommand{\figurename}{Foto} 
		\includegraphics[width=0.8\textwidth]{images/lieber-tanz-ich-als-g20-4}
		\caption{Laura Lagershausen }
	\end{figure}
\end{frame}

	\begin{frame}
	\frametitle{Verlauf friedlicher Demos: 1000 Gestalten (1)}
		05.07.2017. \enquote{1000 Gestalten}\par
		Ca. 1000 Teilnehmer
	\begin{figure}
		\renewcommand{\figurename}{Foto} 
		\includegraphics[width=0.8\textwidth]{images/1000-gestalten-1}
		\caption{1.000 Gestalten / Youtube}
	\end{figure}
	\note[item]{
		Zum 5. Juli hatten 30 Künstler aus Berlin und Hamburg die Kunstperformance \enquote{1000 Gestalten} vorbereitet. Dabei wandelten komplett grau geschminkte und gekleidete Gestalten langsam durch Hamburg-HafenCity und Innenstadt zum Burchardplatz, um \enquote{auf die Auswirkung des Kapitalismus in der jetzigen Form} hinzuweisen. Zuletzt warfen sie die graue Kleidung ab und verwandelten sich in bunten Protest.
	}
\end{frame}

	\begin{frame}
	\frametitle{Verlauf friedlicher Demos: 1000 Gestalten (2)}
	\begin{figure}
		\renewcommand{\figurename}{Foto} 
		\includegraphics[width=0.8\textwidth]{images/1000-gestalten-2}
		\caption{Getty Images, Pool.}
	\end{figure}
\end{frame}

	\begin{frame}
	\frametitle{Verlauf friedlicher Demos: 1000 Gestalten (3)}
	\begin{figure}
		\renewcommand{\figurename}{Foto} 
		\includegraphics[width=0.8\textwidth]{images/1000-gestalten-3}
		\caption{Getty Images, Pool.}
	\end{figure}
\end{frame}

	\begin{frame}
	\frametitle{Verlauf friedlicher Demos: 1000 Gestalten (4)}
	\begin{figure}
		\renewcommand{\figurename}{Foto} 
		\includegraphics[width=0.8\textwidth]{images/1000-gestalten-4}
		\caption{Getty Images, Pool.}
	\end{figure}
\end{frame}

	\begin{frame}
	\frametitle{Verlauf friedlicher Demos: 1000 Gestalten (5)}
	\begin{figure}
		\renewcommand{\figurename}{Foto} 
		\includegraphics[width=0.8\textwidth]{images/1000-gestalten-5}
		\caption{Getty Images, Pool.}
	\end{figure}
\end{frame}

	\begin{frame}
	\frametitle{Verlauf friedlicher Demos: Gipfel für Globale Solidarität (1)}
		05.07.2017. \enquote{Gipfel für Globale Solidarität}\par
		Ca. 1.500 Teilnehmer
	\begin{figure}
		\renewcommand{\figurename}{Foto} 
		\includegraphics[width=0.8\textwidth]{images/gipfel-globale-solidaritaet-1}
		\caption{Harriet Dohmeyer}
	\end{figure}
	\note[item]{
	Der Alternativgipfel ist kostenlos und partizipativ. In elf Podiumsdiskussionen und 75 Workshops werden Teilnehmer aktiv aufgefordert, ihre Ideen einzubringen und zusammenzuarbeiten.
	}
\end{frame}

	\begin{frame}
	\frametitle{Verlauf friedlicher Demos: Gipfel für Globale Solidarität (2)}
	\begin{figure}
		\renewcommand{\figurename}{Foto} 
		\includegraphics[width=0.8\textwidth]{images/gipfel-globale-solidaritaet-2}
		\caption{Harriet Dohmeyer}
	\end{figure}
\end{frame}

	\begin{frame}
	\frametitle{Verlauf friedlicher Demos: Gipfel für Globale Solidarität (3)}
	\begin{figure}
		\renewcommand{\figurename}{Foto} 
		\includegraphics[width=0.8\textwidth]{images/gipfel-globale-solidaritaet-3}
		\caption{Harriet Dohmeyer}
	\end{figure}
\end{frame}

	\begin{frame}
	\frametitle{Verlauf friedlicher Demos: welcome to hell (1)}
	06.07.2017\par 
	Etwa 8000 Demonstranten versammelten sich später zu einer neu angemeldeten Demonstration, die ohne Vorfälle über die vereinbarte Route zog (Spiegel, 7. Juli 2017)
\end{frame}

	\begin{frame}
	\frametitle{Verlauf friedlicher Demos: Grenzenlose Solidarität statt G20 (1)}
		08.07.2017. \enquote{Grenzenlose Solidarität statt G20}\par
		60.000 Teilnehmer
\end{frame}

	\begin{frame}
	\frametitle{Verlauf friedlicher Demos}
	Das waren nicht alle Demos.\par
	\vfill
	\vfill
	Es haben mehrere zehn tausend Demonstranten in Hamburg friedlich demonstriert 
\end{frame}

	\begin{frame}
	\subsection{Nachbereitung}
	\frametitle{Nachbereitung}
	\begin{itemize}
		\item Fazit
		\item Offene Punkte
		\item Dringende Bitte
	\end{itemize}
\end{frame}

\begin{frame}
	\begin{itemize}
		\item G20 nutzt niemandem, außer fotografierten Politikern.
		\item Friedliche Demonstranten müssen geschützt, nicht geknüppelt werden
		\item In einer Demokratie kann die Polizei nicht machen, was sie will
	\end{itemize}
	\vfill
	Gewaltätiger Protest ist kacke, \par
	gewaltätige Polizei ist auch kacke!
\end{frame}

\begin{frame}
	\begin{itemize}
		\item VersammlungsGesetz
		\item Grundrechte
		\item Videos einbinden
		\item Journalisten und Pressefreiheit
		\item Selbstdarstellung der Polizei
		\item Mediale Präsentation
		\item Macht der Sprache (Terrorismus)
		\item Politische Forderungen
		\item Zukunft von Repression 
		\item \dots
	\end{itemize}
\end{frame}

	\begin{frame}
	\subsection{Nachbereitung, Dringende Bitte}
	\frametitle{Nachbereitung, Dringende Bitte}
	Kritisiert die Polizei, fair und ausgewogen.
	\vfill
	\begin{itemize}
		\item Über Kritik an Polizeiverhalten informiert bleiben
		\item Vereine mit Zeit und Geld unterstützen (Rote Hilfe)
		\item Eigene Volksvertreter anschreiben
		\item Eigene Polizeioberhauptüberkommissare anschreiben
	\end{itemize}
	\vfill
	Hasta la victoria siempre!
	\vfill
	Vielen Dank.
	\vfill
	Feedback auch unter https://cfp.mrmcd.net/2017/talk/ZZEFRH/
\end{frame}

\end{document}
